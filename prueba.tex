\documentclass{article}

\usepackage[utf8]{inputenc}

\usepackage{amsmath}

\usepackage[spanish]{babel}


\title{Apuntes de programación lineal}

\author{Nataly Julieth Franco Juárez}

\begin{document}

\maketitle

\tableofcontents

\section{Introducción}
\label{sec:introduccion}



La forma estándar de un problema de programación lineal es:
Dados una matriz $A$ y vectores $b,c$, maximizar $c^tx$ sujeto a
$Ax\leq b$.


La forma simplex de un problema de programación lineal es:
Dado una matriz $A$ y vectores $b,c$, maximizar $c^tx$ sujeto a
$Ax\leq b$.


El método gráfico de un problema de programación lineal es:
Dada una ecuacion $ax+by+c$, maximizar $ax+by+c$ sujeto a
$dx+ey\leq n$ (donde n es una constante) y $fx+gy+h\geq n$

EJEMPLOS:

1: Un gerente está planeando cómo distribuir la produccion de dos
productos entre dos maquinas. Para ser manufacturado cada punto
requiere cierto tiempo (en horas) en cada una de las maquinas. El
tiempo requqerido es resumido en la tabla. La maquina 1 esta
disponible 40 horas a la semana y la maquina2 esta disponible 34 horas
a la semana. Si la utilidad obtenida al vendedor los productos A y B
es de 2, 3 pesos por unidad, respectivamente. ¿Cual será la producción
semanal que maximiza la utilidad?,¿cual es la utilidad maxima?


\begin{tabular}{|c|c|c|}
  \hline
  &A&B\\
  \hline
  maquina1&1&2\\
  \hline
  maquina2&1&1
\end{tabular}


2;
\begin{equation}
  \label{eq:1}
  A=
  \begin{pmatrix}
    0&1&2\\
    0&-1&5
  \end{pmatrix}
  \begin{pmatrix}
    2&3&4\\
    5&6&7
  \end{pmatrix}
\end{equation}



\end{document}