\documentclass{article}

\usepackage[spanish]{babel}
\usepackage{amsmath}
\usepackage[utf8]{inputenc}

\title{El método Simplex}
\author{Nataly Julieth Franco Juárez}

\begin{document}

  \maketitle
  
  \section{Introducción}
  \label{sec:introduccion}

  El método simplex es un algoritmo para resolver problemas de
  programación lineal. Creado por el fisico matemático George Bernad
  Dantzig en el año 1947.

 \section{Ejemplo}
 \label{sec:ejemplo}

 Ilusitramos la aplicación del método simplex con un ejemplo.
\item Resuelve el siguiente problema
  \begin{equation}
    \begin{aligned}
      \text{Maximizar}
      \quad $2x_1+2x_2$\\
      \text{sujeto a}
      \quad &
      \begin{aligned}
        $2x_1+x_2$ \leq 4\\
        $-x_1-x_2$ \geq -5\\
      \end{aligned}
    \end{aligned}
  \end{equation}

  Paso1; Verificar se este problema esta de forma estandar es decir
  que tenga la forma:

  a_{1}+a_{2}+...+a{n}&\leq b\\
   x_1+x_2+...+x_n&\geq 0\\

   \begin{equation}
    \begin{aligned}
      \text{Maximizar} \quad  &2x_1+2x_2 \\
      \text{sujeto a} \quad
       \begin{aligned}
        2x_1+x_2 & \leq 4\\
        -x_1-x_2 & \geq -5\\
      \end{aligned}
      \end{aligned}
  \end{equation}
      



  
\end{document}
